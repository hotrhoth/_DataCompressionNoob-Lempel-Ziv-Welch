\begin{longtable}{|>{\raggedright\arraybackslash}p{8.5cm}|>{\centering\arraybackslash}p{2.5cm}|>{\raggedright\arraybackslash}p{4cm}|}

\caption{Evaluation of Completion Rates for General Project Requirements}

% \hline
% \textbf{Requirement} & \textbf{Completion Rate (\%)} & \textbf{Remarks} \\ \hline
% \endfirsthead

\hline
\textbf{General Requirements} & \textbf{Completion Rate (\%)} & \textbf{Remarks} \\ \hline
\endhead
Group and member information (Student ID, full name, etc.) & 100 & Information provided completely and accurately. \\ \hline
Work assignment table, which includes information on each task assigned to team members, along with the completion rate of each member compared to the assigned tasks. & 100 & All tasks and contributions documented in detail. \\ \hline
Self-evaluation of the completion rate of the project requirements. & 100 & Honest and thorough self-assessment included. \\ \hline
Detailed description of each algorithm. Illustrative images and diagrams are encouraged. & 100 & Algorithms are well-explained with clear tables and examples. \\ \hline
Description of the test cases and experiment results along with your insights. & 100 & Test cases and results thoroughly analyzed with meaningful insights provided. \\ \hline
The report needs to be well-formatted and exported to PDF. If there are figures cut-off by the page break, etc., points will be deducted. & 100 & Formatting is clean and all figures are properly aligned without any cut-off. \\ \hline
References & 100 & Complete and correctly formatted reference list included. \\ \hline
\end{longtable}

\newpage

\begin{longtable}{|>{\raggedright\arraybackslash}p{8.5cm}|>{\centering\arraybackslash}p{2.5cm}|>{\raggedright\arraybackslash}p{4cm}|}

\caption{Evaluation of Completion Rates for Project LZW Compression Requirements}

% \hline
% \textbf{Requirement} & \textbf{Completion Rate (\%)} & \textbf{Remarks} \\ \hline
% \endfirsthead

\hline
\textbf{Requirements} & \textbf{Completion Rate (\%)} & \textbf{Remarks} \\ \hline
\endhead
Introduce its background: history and applications. & 100 & Background and applications of the algorithm are thoroughly explored, providing clear historical context and modern use cases. \\ \hline
Trace the algorithm, step by step, for both compression and decompression phases, using simple examples. & 100 & Step-by-step tracing includes clear examples for both phases, enhancing understanding. \\ \hline
Analyze the algorithm’s time complexity in the best case and worst case. & 100 & Comprehensive analysis provided with detailed calculations for both scenarios. \\ \hline
Present a comprehensive overview of LZW-related algorithms, such as the LZ77 and LZ78. For each algorithm, just give the overall idea while ignoring the details. & 100 & LZW-related algorithms are briefly but effectively summarized, focusing on key concepts. \\ \hline
\end{longtable}